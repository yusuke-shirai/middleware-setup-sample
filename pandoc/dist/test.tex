\documentclass[11pt,a4j]{ltjsarticle}
\usepackage{spec}
\usepackage{lmodern}
\usepackage{amssymb,amsmath}
\usepackage{ifxetex,ifluatex}
\usepackage{fixltx2e} % provides \textsubscript
\ifnum 0\ifxetex 1\fi\ifluatex 1\fi=0 % if pdftex
  \usepackage[T1]{fontenc}
  \usepackage[utf8]{inputenc}
\else % if luatex or xelatex
  \usepackage{unicode-math}
  \defaultfontfeatures{Ligatures=TeX,Scale=MatchLowercase}
\fi
% use upquote if available, for straight quotes in verbatim environments
\IfFileExists{upquote.sty}{\usepackage{upquote}}{}
% use microtype if available
\IfFileExists{microtype.sty}{%
\usepackage[]{microtype}
\UseMicrotypeSet[protrusion]{basicmath} % disable protrusion for tt fonts
}{}
\PassOptionsToPackage{hyphens}{url} % url is loaded by hyperref
\usepackage[unicode=true]{hyperref}
\hypersetup{
            pdftitle={サンプルアプリケーション機能仕様書},
            pdfborder={0 0 0},
            breaklinks=true}
\urlstyle{same}  % don't use monospace font for urls
\usepackage{longtable,booktabs}
% Fix footnotes in tables (requires footnote package)
\IfFileExists{footnote.sty}{\usepackage{footnote}\makesavenoteenv{long table}}{}
\usepackage{graphicx,grffile}
\makeatletter
\def\maxwidth{\ifdim\Gin@nat@width>\linewidth\linewidth\else\Gin@nat@width\fi}
\def\maxheight{\ifdim\Gin@nat@height>\textheight\textheight\else\Gin@nat@height\fi}
\makeatother
% Scale images if necessary, so that they will not overflow the page
% margins by default, and it is still possible to overwrite the defaults
% using explicit options in \includegraphics[width, height, ...]{}
\setkeys{Gin}{width=\maxwidth,height=\maxheight,keepaspectratio}
\IfFileExists{parskip.sty}{%
\usepackage{parskip}
}{% else
\setlength{\parindent}{0pt}
\setlength{\parskip}{6pt plus 2pt minus 1pt}
}
\setlength{\emergencystretch}{3em}  % prevent overfull lines
\providecommand{\tightlist}{%
  \setlength{\itemsep}{0pt}\setlength{\parskip}{0pt}}
\setcounter{secnumdepth}{5}
% Redefines (sub)paragraphs to behave more like sections
\ifx\paragraph\undefined\else
\let\oldparagraph\paragraph
\renewcommand{\paragraph}[1]{\oldparagraph{#1}\mbox{}}
\fi
\ifx\subparagraph\undefined\else
\let\oldsubparagraph\subparagraph
\renewcommand{\subparagraph}[1]{\oldsubparagraph{#1}\mbox{}}
\fi

% set default figure placement to htbp
\makeatletter
\def\fps@figure{htbp}
\makeatother


\title{サンプルアプリケーション機能仕様書}
\date{\西暦{\today}}

\TPM{TPM太郎}
\GTPM{GTPM太郎}
\PTL{PTL太郎}
\GPM{GPM太郎}

\begin{document}
\maketitle

{
\setcounter{tocdepth}{3}
\tableofcontents
}
\section{はじめに}\label{ux306fux3058ux3081ux306b}

本書では、サンプルのアプリケーションの機能仕様を説明します。

\subsection{本システムの位置付け}\label{ux672cux30b7ux30b9ux30c6ux30e0ux306eux4f4dux7f6eux4ed8ux3051}

このアプリケーションは、仕様書のテンプレート説明のための架空のアプリケーションです。

\subsection{用語の定義}\label{ux7528ux8a9eux306eux5b9aux7fa9}

\begin{description}
\item[HTML(ハイパーテキスト マークアップ
ランゲージ)、HTML(エイチティーエムエル)]
ハイパーテキストを記述するためのマークアップ言語の1つである。
\item[Markdown(マークダウン)]
文書を記述するための軽量マークアップ言語のひとつである。
\end{description}

\section{動作環境}\label{ux52d5ux4f5cux74b0ux5883}

\subsection{サーバー動作環境}\label{ux30b5ux30fcux30d0ux30fcux52d5ux4f5cux74b0ux5883}

\begin{longtable}[c]{@{}ll@{}}
\toprule\addlinespace
項目 & 値
\\\addlinespace
\midrule\endhead
オペレーティングシステム & CentOS 7
\\\addlinespace
データベース & PostgreSQL 9.6
\\\addlinespace
Java & Java 8
\\\addlinespace
\bottomrule
\end{longtable}

\subsection{クライアント動作環境}\label{ux30afux30e9ux30a4ux30a2ux30f3ux30c8ux52d5ux4f5cux74b0ux5883}

\begin{longtable}[c]{@{}ll@{}}
\toprule\addlinespace
項目 & 値
\\\addlinespace
\midrule\endhead
オペレーティングシステム & Windows 10
\\\addlinespace
ブラウザ & Google Chrome
\\\addlinespace
\bottomrule
\end{longtable}

\section{機能の概要}\label{function}

\subsection{ユーザー向け機能}\label{ux30e6ux30fcux30b6ux30fcux5411ux3051ux6a5fux80fd}

ユーザーは主に以下の流れで本システムを利用すると想定する。

\begin{figure}[htbp]
\centering
\includegraphics{img/test.png}
\caption{利用の流れ}
\end{figure}

\hyperdef{}{function-login}{\subsubsection{ログイン}\label{function-login}}

ログイン機能の説明

\subsubsection{登録}\label{function-create}

登録することができます。

\paragraph{前提条件}\label{ux524dux63d0ux6761ux4ef6}

ログインしている必要があります。ログインの詳細は、\hyperref[function-login]{ログイン}を参照のこと。

\paragraph{入力}\label{ux5165ux529b}

以下の入力が必要です。(順序性はありませんが、あえて番号付きリストで書きます。)

\begin{enumerate}
\def\labelenumi{\arabic{enumi}.}
\itemsep1pt\parskip0pt\parsep0pt
\item
  登録先
\item
  登録するデータ
\end{enumerate}

\paragraph{エラーケース}\label{ux30a8ux30e9ux30fcux30b1ux30fcux30b9}

以下の場合は \textbf{エラー} になります。

\begin{itemize}
\itemsep1pt\parskip0pt\parsep0pt
\item
  ログインしていない場合
\item
  登録データが空の場合
\end{itemize}

\subsubsection{削除}\label{function-delete}

削除をすることができます。

\subsection{管理者向け機能}\label{ux7ba1ux7406ux8005ux5411ux3051ux6a5fux80fd}

ユーザーを登録削除する機能を保持します。

\end{document}
